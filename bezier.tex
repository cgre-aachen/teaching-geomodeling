\documentclass{article}
\usepackage{amsmath}

\begin{document}

\section*{Quadratic Bézier Curve}

A quadratic Bézier curve is defined by three points: \( P_0 \), \( P_1 \), and \( P_2 \). The curve is defined by the parameter \( t \) which ranges from 0 to 1.

\textbf{Mathematical Formulation:}
\[
B(t) = (1-t)^2 \cdot P_0 + 2(1-t) \cdot t \cdot P_1 + t^2 \cdot P_2
\]

\subsection*{Derivation:}

\begin{enumerate}
    \item We start with linear interpolations between each pair of control points:
    \begin{align*}
        \text{Between } P_0 \text{ and } P_1 &: \quad Q_0(t) = (1-t) \cdot P_0 + t \cdot P_1 \\
        \text{Between } P_1 \text{ and } P_2 &: \quad Q_1(t) = (1-t) \cdot P_1 + t \cdot P_2
    \end{align*}
    
    \item Now, interpolate between \( Q_0(t) \) and \( Q_1(t) \):
    \[
    B(t) = (1-t) \cdot Q_0(t) + t \cdot Q_1(t)
    \]
    
    \item Substituting in our expressions for \( Q_0(t) \) and \( Q_1(t) \) from step 1:
    \[
    B(t) = (1-t) \left[ (1-t) \cdot P_0 + t \cdot P_1 \right] + t \left[ (1-t) \cdot P_1 + t \cdot P_2 \right]
    \]
    
    \item Expanding and grouping terms:
    \[
    B(t) = (1-t)^2 \cdot P_0 + 2(1-t) \cdot t \cdot P_1 + t^2 \cdot P_2
    \]
\end{enumerate}

\end{document}
