\documentclass{article}
\usepackage{amsmath}
\begin{document}

\section*{Polynomial Interpolation}

\textbf{Concept}: Polynomial interpolation is the process of constructing a polynomial that passes exactly through a given set of data points. Assume you have \( n + 1 \) distinct data points \( (x_0, y_0), (x_1, y_1), \ldots, (x_n, y_n) \). The objective is to find a polynomial \( P(x) \) of degree at most \( n \) that satisfies \( P(x_i) = y_i \) for each \( i = 0, 1, \ldots, n \).

\textbf{Mathematical Formulation}:

\begin{enumerate}
    \item \textbf{Polynomial Representation}:
    The polynomial \( P(x) \) is generally represented as:
    \[
    P(x) = a_0 + a_1 x + a_2 x^2 + \ldots + a_n x^n
    \]
    Here, \( a_0, a_1, \ldots, a_n \) are the coefficients that we need to determine.

    \item \textbf{Lagrange Interpolation Formula}:
    An often-used method for finding these coefficients is the Lagrange interpolation formula, where:
    \[
    P(x) = \sum_{i=0}^{n} y_i \cdot L_i(x)
    \]
    In this formula, \( L_i(x) \) are the Lagrange basis polynomials, defined as:
    \[
    L_i(x) = \prod_{\substack{j=0 \\ j \neq i}}^{n} \frac{x - x_j}{x_i - x_j}
    \]
    Each \( L_i(x) \) is a polynomial that equals 1 at \( x = x_i \) and 0 at all other \( x_j \) points.
\end{enumerate}

This approach ensures that the resulting polynomial \( P(x) \) passes through all the provided data points.

\section*{Polynomial Splines}

\textbf{Concept}: Polynomial splines are piecewise polynomial functions that are used to approximate a smooth curve through a set of data points. They are particularly useful for modeling data that is not well-represented by a single polynomial function. Cubic splines are a common type of spline, consisting of piecewise cubic polynomials.

\textbf{Cubic Splines}:

Assume you have a set of \( n+1 \) data points \( (x_0, y_0), (x_1, y_1), \ldots, (x_n, y_n) \). A cubic spline is a function \( S(x) \) that satisfies the following conditions:

\begin{enumerate}
    \item On each interval \([x_i, x_{i+1}]\), \( S(x) \) is a cubic polynomial. Denote these polynomials as \( S_i(x) \) for \( i = 0, 1, \ldots, n-1 \).
    \item \( S(x_i) = y_i \) for all \( i = 0, 1, \ldots, n \), ensuring the spline passes through all the data points.
    \item The first and second derivatives of \( S(x) \) are continuous across the intervals. This means for each \( i = 1, 2, \ldots, n-1 \), we have:
    \[
    S_{i-1}'(x_i) = S_i'(x_i) \quad \text{and} \quad S_{i-1}''(x_i) = S_i''(x_i)
    \]
    \item Additional boundary conditions are required to uniquely determine the spline. Common choices are to set \( S''(x_0) = 0 \) and \( S''(x_n) = 0 \) (natural spline), or to specify the first derivatives at the endpoints.
\end{enumerate}

Each cubic polynomial \( S_i(x) \) can be written in the form:
\[
S_i(x) = a_i + b_i(x - x_i) + c_i(x - x_i)^2 + d_i(x - x_i)^3
\]
The coefficients \( a_i, b_i, c_i, \) and \( d_i \) are determined by the conditions above and by ensuring the continuity of the spline across its intervals.

Cubic splines are widely used for interpolation and approximation due to their smoothness and flexibility in fitting data.


\end{document}
